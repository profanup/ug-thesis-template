% Chapter Template

\chapter{REQUIREMENTS AND ANALYSIS} % Main chapter title

\label{Chapter4} % Change X to a consecutive number; for referencing this chapter elsewhere, use \ref{ChapterX}

\lhead{Chapter 4. \emph{REQUIREMENTS AND ANALYSIS}} % Change X to a consecutive number; this is for the header on each page - perhaps a shortened title

%----------------------------------------------------------------------------------------
%	SECTION 1
%----------------------------------------------------------------------------------------

\section{Problem Definition}
Define the problem on which you are working in the project. Provide details of the overall problem and then divide the problem in to sub-problems.
Define each sub-problem clearly.

\section{Requirements Specification}
In this phase you should define the requirements of the system, independent of how these requirements will be accomplished. The Requirements Specification describes the things in the system and the actions that can be done on these things. Identify the operation and problems of the existing system.

\section{Planning and Scheduling}
Planning and scheduling is a complicated part of software development. Planning, for our purposes, can be thought of as determining all the small tasks that must be carried out in order to accomplish the goal. Planning also takes into account, rules, known as constraints, which, control when certain tasks can or cannot happen. Scheduling can be thought of as determining whether adequate resources are available to carry out the plan. You should show the Gantt chart and Program Evaluation Review Technique (PERT).

\section{Software and Hardware Requirements}
Define the details of all the software and hardware needed for the development and implementation of your project.

\textit{Hardware Requirement: In this section, the equipment, graphics card, numeric co-processor, mouse, disk capacity, RAM capacity etc. necessary to run the software must be noted.}

\textit{Software Requirements: In this section, the operating system, the compiler, testing tools, linker, and the libraries etc. necessary to compile, link and install the software must be listed.}

\section{Preliminary Product Description}
Identify the requirements and objectives of the new system. Define the functions and operation of the application/system you are developing as your project.

\section{Conceptual Models}
You should understand the problem domain and produce a model of the system, which describes operations that can be performed on the system, and the allowable sequences of those operations. Conceptual Models could consist of complete Data Flow Diagrams, ER diagrams, Object-oriented diagrams, System Flowcharts etc.
