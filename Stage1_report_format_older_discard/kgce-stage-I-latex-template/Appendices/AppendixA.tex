% Appendix Template

\chapter{Appendix A} % Main appendix title

\label{AppendixA} % Change X to a consecutive letter; for referencing this appendix elsewhere, use \ref{AppendixX}

\lhead{Appendix A. \emph{Appendix Title Here}} % Change X to a consecutive letter; this is for the header on each page - perhaps a shortened title

Write your Appendix content here.....These may be provided to include further details of results, mathematical derivations, certain illustrative parts of the program code (e.g., class interfaces), user documentation etc.

\textbf{\Large PROJECT REPORT STRUCTURE}

\textbf{INTRODUCTION}

The project report should be documented with an engineering approach to the solution of the problem that you have sought to address. The project report should be prepared in order to solve the problem in a methodical and professional manner, making due references to appropriate techniques, technologies and professional standards. You should start the documentation process from the first step of software development so that you can easily identify the issues to be focused upon in the ultimate project report. You should also include the details from your project notebook, in which you would have recorded the progress of your project throughout the course. The project report should contain enough details to enable examiners to evaluate your work. The
details, however, should not render your project report as boring and tedious. The important points should be highlighted in the body of the report, with details often
relegated to appendices

\textbf{IMPORTANCE OF PROJECT/PROJECT REPORT}

The Mini Project is not only a part of the coursework, but also a mechanism to demonstrate your abilities and specialisation. It provides the opportunity for you to demonstrate originality, teamwork, inspiration, planning and organisation in a software project, and to put into practice some of the techniques you have been taught throughout the previous courses. The Mini Project is important for a number of reasons. It provides students with:

\begin{enumerate}
\item[$\bullet$]opportunity to specialise in specific areas of IT;
\item[$\bullet$]future employers will most likely ask you about your project at interview;
\item[$\bullet$]opportunity to demonstrate a wide range of skills and knowledge learned, and
\item[$\bullet$]encourages integration of knowledge gained in the previous course units.
\end{enumerate}


\textbf{The project report is an extremely important aspect of the project. It serves to show what you have achieved and should demonstrate that:}

\textbf{\textit{Elements of Project Development}}
\begin{enumerate}


\item[$\bullet$]You understand the wider context of computing by relating your choice of the
project, and the approach you take, to existing products or research.

\item[$\bullet$]You can apply the theoretical and practical techniques taught in the course to
the problem you are addressing and that you understand their relevance to the
wider world of computing.

\item[$\bullet$]You are capable of objectively criticising your own work and making constructive suggestions for improvements or further work based on your experiences so far.

\item[$\bullet$]You can explain your thinking and working processes clearly and concisely to others through your project report.

\end{enumerate}
