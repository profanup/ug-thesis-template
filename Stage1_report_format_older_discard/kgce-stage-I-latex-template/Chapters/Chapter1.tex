% Chapter Template

\chapter{INTRODUCTION} % Main chapter title

\label{Chapter1} % Change X to a consecutive number; for referencing this chapter elsewhere, use \ref{ChapterX}

\lhead{Chapter 1. \emph{INTRODUCTION}} % Change X to a consecutive number; this is for the header on each page - perhaps a shortened title

%----------------------------------------------------------------------------------------
%	SECTION 1
%----------------------------------------------------------------------------------------

\section{Introduction}
The introduction has several parts as given below:
Background: A description of the background and context of the project and its relation to work already done in the area. Summarise existing work in the area concerned with your project work.

\section{Objectives}
Objectives: Concise statement of the aims and objectives of the project. Define exactly what you are going to do in the project; the objectives should be about 30 /40 words.

\section{Purpose, Scope, and Applicability}
Purpose, Scope and Applicability: The description of Purpose, Scope, and Applicability are given below:

\subsection{Purpose}
Purpose: Description of the topic of your project that answers questions on why you are doing this project. How your project could improve the system its significance and theoretical framework.

\subsection{Scope}
Scope: A brief overview of the methodology, assumptions and limitations. You should answer the question: What are the main issues you are covering in your project? What are the main functions of your project?

\subsection{Applicability}
Applicability: You should explain the direct and indirect applications of your work. Briefly discuss how this project will serve the computer world and people.

\section{Achievements}
Achievements: Explain what knowledge you achieved after the completion of your work. What contributions has your project made to the chosen area? Goals achieved describe the degree to which the findings support the original objectives laid out by the project. The goals may be partially or fully achieved, or exceeded.

\section{Organisation of Report}
Organization of Report: Summarizing the remaining chapters of the project report,
in effect, giving the reader an overview of what is to come in the project report.






